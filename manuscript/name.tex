% Bibliography

% Uncomment the next line (\nocite{*}) if you want to include all items from your .bib file
% (e.g. if you didn't use the \textcite or \parencite commands above)
% \nocite{*}

% This command generates the bibliography


% Include the body
% Note: You can also type it directly here, or include a file per section, whatever works best for you

Regular text is written as plain text. 
You can write an entire paragraph in one line, 
or split lines in any way you want

\subsection{Using Latex markup commands}

As you can see, markup such as (sub)section headers are introduced using latex commands starting with a backslash.
You can use \textit{italic} and \textbf{bold} text as needed. 

Quotations can be given latex style for `single' or ``double'' quotations marks,
or a bit easier using the "csquotes" packages included above.
You can also copy-paste the ‘smart quotes’ directly from e.g. MS Word.

\subsection{Tables, figures, lists and cross-references}

See Table~\ref{tab:my_label} and \ref{fig:my_label} for an example table and figure.
You can learn more about tables and figures in latex from the overleaf documentation at
\url{https://www.overleaf.com/learn/latex/Tables} and \url{https://www.overleaf.com/learn/latex/Inserting_Images}.

For tables, we believe that the best results are achieved by using the `booktabs' and `tabularx' packages,
and following some simple rules, shown in the numbered ("enumerated") list below:
\begin{enumerate}
    \item Follow APA rules for tables
    \item Make all tables full-width
    \item Never using vertical lines
    \item Using horizontal lines very 
\end{enumerate}

% Note that in latex, tables and figures are 'floats' and will not be set directly where 
% you type them, but in a location that matches the general style, usually at the top of a page
% You can include a position hint such as the 't' below, other options are 'b' (bottom) and 'h' (here)
\begin{table}[t] 
    \centering
    % Note: Xlcr in the line below specifies 3 columns with 'X', 'l', 'c', 'r', and 'S' as column types.  
    % The first column (X) is for teXt, and will be expanded to fill the width of the table
    % The other three columns will be just wide enough to fit the contents, 
    % and will be aligned left, center, and right respectively
    % The last column uses the 'S' type from the 'siunitx' package, which aligns on decimal marks
    % Note that if you want to have regular text in such columns (e.g. in the header), 
    % you need to enclose it in {curly braces}
    \begin{tabularx}{\textwidth}{XlcrS}
    % Table rows use & to separate cells and end with \\
    % Use \toprule, \midrule and \bottomrule for the horizontal lines
    \toprule
       First header & Left & Center & Right & {Decimal} \\
    \midrule
    Description of the first row & 3.14 & A & 2.19 & 2.19\\
    Description of the second row & 10.18 & & 10.08 & 10.1\\
    \bottomrule
    \end{tabularx}
    \caption{The caption of your table}
    % Label for cross-references, you can use any name here that you want
    \label{tab:my_label}
\end{table}

\begin{figure}[b]
    \centering
    \includegraphics[width=.6\textwidth]{aup_logo.pdf}
    \caption{Caption of your figure}
    \label{fig:my_label}
\end{figure}


\section{Citations}

References should be included in  ‘bibtex’ format. 
You can easily export bibtex files from endnote, zotero etc. and upload them in Overleaf.

To generate in-text citations, you can use the `textcite'  \textcite{ccrintro} said: ``CCR is great'' \parencite[p.~1]{ccrintro}
